\documentclass[10pt,conference,compsocconf]{IEEEtran}

\usepackage{hyperref}
\usepackage{graphicx}	% For figure environment


\begin{document}
\title{Class Project 1 Report}

\author{
  Angerand Gr\'egoire, Goullet Boris, Grondier Julien 
}

\maketitle

\begin{abstract}
% Present the project here I guess ? Say our implementation uses such and such method.
\end{abstract}

\section{Introduction}
% Do we even need one ?

\section{Models and Methods}
% started by looking at the dataset ; realized 4 buckets ; applied linreg ; applied ridge ; tried logit but shit results
\subsection{Exploratory Data Analysis}
First, we started off this project by opening up the dataset in a spreadsheet software.

We quickly realized that a number of columns were sometimes undefined (-999). However, we also noted that, apart from DER_mass_MMC, which features were defined was completely determined by the value column 22, PRI_jet_num, which, incidentally, is the only categorical column. %TODO: explain which columns are defined for which pri jet num

We also plotted the distribution of each feature, relative to whether the event was categorized as a HB. These plots can be found in data/images/. They show that some features have a normal distribution (which is good for linear regression), but others seem to have an inverse %what's the actual term ?
or parabolic distribution.

\subsection{Feature processing}
Since we wanted versions of features with normal distributions, we decided to construct artificial features from existing ones and add them as feature columns. As such, our prepare_data method adds inverse, log, and polynomial versions of the individual feature columns.

We also looked at the pearson correlation between the different features and whether or not they were a Higgs Boson. By removing the %how many ?
columns with the lowest PCC (less than 0.01) we saved ourselves a lot of processing time and RAM usage, while lowering our cross-validated precision on the training set by less than 0.5\%.

\section{Results}
% shit

\section{Discussion}
%?


\section{Summary}



\end{document}
