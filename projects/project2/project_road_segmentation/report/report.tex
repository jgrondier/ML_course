\documentclass[11pt,conference,compsocconf]{IEEEtran}

\usepackage{hyperref}
\usepackage{graphicx}	% For figure environment
\usepackage[english]{babel}
\usepackage{tabu}
\usepackage{xargs}  
\usepackage[prependcaption,textsize=tiny]{todonotes}
\presetkeys{todonotes}{inline,backgroundcolor=red}{}


\begin{document}
\title{Class Project 2 Report: Road Segmentation Challenge}

\author{
  Angerand Gr\'egoire, Goullet Boris, Grondier Julien 
}

\maketitle

\begin{abstract}
In this project, we were given a dataset consisting of aerial photographs of suburban areas. Our task was to apply different machine learning methods to this dataset to train a model that would predict where roads were present in these photographs.

\todo{summarize results}
\end{abstract}


\section{Models applied}

\subsection{Logistic regression}
At first, we tried using a logistic regression. We did this by cutting up the images into small patches (either 4x4, 8x8, or 16x16) and training a logist model on 3 features for that patch: mean Red pixels values, mean Green pixel values, and mean Blue pixel values.

However, results were less than satisfactory, even with more/different features or patch sizes, so we quickly dropped that idea.

\subsection{Convolutional Neural Network}
Started off by using the given tensorflow network

\subsubsection{Custom CNN classifier}


\section{Feature processing}

We tried adding house labels, but didn't help.

We train on 500 images instead of 100: the extra 400 were transposed, flipped each side and both at the same time.


\section{Results post-processing}
After obtaining the raw prediction from the convolutional neural network, we do a post-processing step in order to remove most of the noise.

The post processing consists of a series of convolutions, using a cross shaped kernel. This shape has be chosen to match the many right angle intersections present in our dataset.

In some cases the roads are not aligned to the image axes, to handle those we extract the most significant straight lines from the image using Hough transform and rotate our kernel to match their orientation.


\section{Discussion}


\section{Summary}

\end{document}
